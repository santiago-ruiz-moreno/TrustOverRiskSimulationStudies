% Options for packages loaded elsewhere
\PassOptionsToPackage{unicode}{hyperref}
\PassOptionsToPackage{hyphens}{url}
%
\documentclass[
]{article}
\usepackage{amsmath,amssymb}
\usepackage{iftex}
\ifPDFTeX
  \usepackage[T1]{fontenc}
  \usepackage[utf8]{inputenc}
  \usepackage{textcomp} % provide euro and other symbols
\else % if luatex or xetex
  \usepackage{unicode-math} % this also loads fontspec
  \defaultfontfeatures{Scale=MatchLowercase}
  \defaultfontfeatures[\rmfamily]{Ligatures=TeX,Scale=1}
\fi
\usepackage{lmodern}
\ifPDFTeX\else
  % xetex/luatex font selection
\fi
% Use upquote if available, for straight quotes in verbatim environments
\IfFileExists{upquote.sty}{\usepackage{upquote}}{}
\IfFileExists{microtype.sty}{% use microtype if available
  \usepackage[]{microtype}
  \UseMicrotypeSet[protrusion]{basicmath} % disable protrusion for tt fonts
}{}
\makeatletter
\@ifundefined{KOMAClassName}{% if non-KOMA class
  \IfFileExists{parskip.sty}{%
    \usepackage{parskip}
  }{% else
    \setlength{\parindent}{0pt}
    \setlength{\parskip}{6pt plus 2pt minus 1pt}}
}{% if KOMA class
  \KOMAoptions{parskip=half}}
\makeatother
\usepackage{xcolor}
\usepackage[margin=1in]{geometry}
\usepackage{graphicx}
\makeatletter
\newsavebox\pandoc@box
\newcommand*\pandocbounded[1]{% scales image to fit in text height/width
  \sbox\pandoc@box{#1}%
  \Gscale@div\@tempa{\textheight}{\dimexpr\ht\pandoc@box+\dp\pandoc@box\relax}%
  \Gscale@div\@tempb{\linewidth}{\wd\pandoc@box}%
  \ifdim\@tempb\p@<\@tempa\p@\let\@tempa\@tempb\fi% select the smaller of both
  \ifdim\@tempa\p@<\p@\scalebox{\@tempa}{\usebox\pandoc@box}%
  \else\usebox{\pandoc@box}%
  \fi%
}
% Set default figure placement to htbp
\def\fps@figure{htbp}
\makeatother
\setlength{\emergencystretch}{3em} % prevent overfull lines
\providecommand{\tightlist}{%
  \setlength{\itemsep}{0pt}\setlength{\parskip}{0pt}}
\setcounter{secnumdepth}{-\maxdimen} % remove section numbering
\usepackage{booktabs}
\usepackage{booktabs}
\usepackage{longtable}
\usepackage{array}
\usepackage{multirow}
\usepackage{wrapfig}
\usepackage{float}
\usepackage{colortbl}
\usepackage{pdflscape}
\usepackage{tabu}
\usepackage{threeparttable}
\usepackage{threeparttablex}
\usepackage[normalem]{ulem}
\usepackage{makecell}
\usepackage{xcolor}
\usepackage{bookmark}
\IfFileExists{xurl.sty}{\usepackage{xurl}}{} % add URL line breaks if available
\urlstyle{same}
\hypersetup{
  pdftitle={Simulations of Trust Game},
  pdfauthor={J. Santiago Ruiz M},
  hidelinks,
  pdfcreator={LaTeX via pandoc}}

\title{Simulations of Trust Game}
\author{J. Santiago Ruiz M}
\date{2025-07-08}

\begin{document}
\maketitle

\section{Simulations of Trust over Risk Game with LLM
Interventions}\label{simulations-of-trust-over-risk-game-with-llm-interventions}

This document presents a simulation study of a repeated trust game,
designed to evaluate the effects of different
treatments---LLMdelegation, LLMadvice, and control---on principal trust,
agent risk-taking, and punishment behavior. The simulation models a
scenario where, in each round, a principal decides whether to trust an
agent, and the agent then chooses between a safe action (``keep'') and a
risky action (``Gamble A''). The probabilities of these actions, as well
as the expected punishment, are parameterized by treatment. This
document concerns only the main hypotheses without the mediators, check
the rmd file to see the code and the simulation setup.

The simulation is based on several key assumptions regarding the
distributions of actions and the expected effect sizes as described in
Table 1. For instance, we anticipate for the agent's choice, the
probability of selecting the riskier ``Gamble A'' to be 1.0 for
LLMdelegation, 0.8 for LLMadvice, and 0.5 for the baseline.

\begin{itemize}
\item
  \textbf{Principal Trust}: The probability that a principal chooses to
  trust the agent is modeled as a function of the agent's treatment.
  Formally, the hypothesis is tested using a logistic mixed effects
  model:

  \[
  Trust_{ij} = \beta_0 + \beta_1 \cdot \text{LLMdelegation}_{ij} + \beta_2 \cdot \text{LLMadvice}_{ij} + u_j
  \] where \(u_j\) is a random intercept for each principal.
\item
  \textbf{Agent Riskier Action}: The likelihood that the agent chooses
  the riskier action (``Gamble A'') is also modeled as a function of
  treatment, using a logistic mixed effects model:

  \[
  \text{logit}(\Pr(\text{GambleA}_{ij} = 1)) = \gamma_0 + \gamma_1 \cdot \text{LLMdelegation}_{ij} + \gamma_2 \cdot \text{LLMadvice}_{ij} + v_j
  \] where \(v_j\) is a random intercept for each agent.
\item
  \textbf{Punishment}: The amount of punishment assigned by the
  principal is modeled with a linear mixed effects model:

  \[
  \text{Punishment}_{ij} = \alpha_0 + \alpha_1 \cdot \text{LLMdelegation}_{ij} + \alpha_2 \cdot \text{LLMadvice}_{ij} + w_j + \epsilon_{ij}
  \] where \(w_j\) is a random intercept for each agent and
  \(\epsilon_{ij}\) is the residual error.
\end{itemize}

The simulation iterates over multiple sample sizes and repetitions to
estimate the statistical power for detecting treatment effects in each
model. The results are summarized in regression tables and power curves,
providing guidance for experimental design and sample size planning.

Participants are rematched in groups of 2, with each group playing 3
rounds of the trust over risk game. We made the following assumptions
about the estimated effect sizes of the experimnt:

Below are the key simulation parameters for each treatment group. The
table summarizes the assumed probabilities and means (\(p\), \(\mu\)),
as well as standard deviations (\(\sigma\)) for trust, agent actions,
and punishment.

\begin{table}[ht]
\centering
\begin{tabular}{lccc}
\toprule
 & \textbf{LLMdelegation} & \textbf{LLMadvice} & \textbf{Control} \\
\midrule
Principal trust probability ($p_\text{trust}$) & 0.8 & 0.6 & 0.5 \\
Mean trust sent ($\mu_\text{trust}$) & 6 & 6 & 5 \\
SD trust sent ($\sigma_\text{trust}$) & 3.2 & 3.2 & 3.2 \\
Agent "keep" probability ($p_\text{keep}$) & 0.05 & 0.05 & 0.05 \\
Agent "Gamble A" probability ($p_\text{GambleA}$) & 1.0 & 0.8 & 0.5 \\
Mean punishment ($\mu_\text{punish}$) & 0 & 0 & 2 \\
SD punishment ($\sigma_\text{punish}$) & 2 & 2 & 2 \\
\bottomrule
\end{tabular}
\caption{Simulation parameters for each treatment group: probabilities ($p$), means ($\mu$), and standard deviations ($\sigma$).}
\end{table}

\begin{table}[!htbp] \centering 
  \caption{Mixed Effects Regression Results} 
  \label{} 
\begin{tabular}{@{\extracolsep{5pt}}lccc} 
\\[-1.8ex]\hline 
\hline \\[-1.8ex] 
 & \multicolumn{3}{c}{\textit{Dependent variable:}} \\ 
\cline{2-4} 
\\[-1.8ex] & Principal Trust (logit) & Riskier Action by Agent (logit) & Punishment (linear) \\ 
\\[-1.8ex] & \textit{linear} & \textit{generalized linear} & \textit{linear} \\ 
 & \textit{mixed-effects} & \textit{mixed-effects} & \textit{mixed-effects} \\ 
 & Principal Trust & Riskier Action & Punishment \\ 
\hline \\[-1.8ex] 
 LLMdelegation & 0.626 & 3.706$^{***}$ & $-$0.939$^{***}$ \\ 
  & & & \\ 
 LLMadvice & 0.630 & 1.517$^{***}$ & $-$0.878$^{***}$ \\ 
  & & & \\ 
 Constant & 5.235$^{***}$ & $-$0.272 & 0.819$^{***}$ \\ 
  & & & \\ 
\hline \\[-1.8ex] 
Observations & 450 & 287 & 450 \\ 
Log Likelihood & $-$1,112.699 & $-$113.579 & $-$889.234 \\ 
Akaike Inf. Crit. & 2,235.399 & 235.157 & 1,788.467 \\ 
Bayesian Inf. Crit. & 2,255.945 & 249.795 & 1,809.014 \\ 
\hline 
\hline \\[-1.8ex] 
\textit{Note:}  & \multicolumn{3}{r}{$^{*}$p$<$0.05; $^{**}$p$<$0.01; $^{***}$p$<$0.001} \\ 
 & \multicolumn{3}{r}{Standard errors in parentheses. * p<0.05, ** p<0.01, *** p<0.001} \\ 
\end{tabular} 
\end{table}

\section{Simulation Results}\label{simulation-results}

\textbar{} \textbar{} \textbar{} 0\%Simulation cache already exists.
Skipping simulation and using cached results.

\begin{table}[!h]

\caption{\label{tab:plot_power}Estimated Power for Hypothesis Tests of Treatment Effects: $\beta_1$, $\beta_2$ (Principal Trust); $\gamma_1$, $\gamma_2$ (Riskier Action); $\alpha_1$, $\alpha_2$ (Punishment)}
\centering
\fontsize{10}{12}\selectfont
\begin{tabular}{llrr}
\toprule
Sample Size & Model & $\beta_1$, $\gamma_1$, $\alpha_1$ (LLMdelegation) & $\beta_2$, $\gamma_2$, $\alpha_2$ (LLMadvice)\\
[1.5ex]
\midrule
90 & Principal Trust & 0.52 & 0.54\\
[1.5ex]
 & Punishment & 0.98 & 0.97\\
[1.5ex]
 & Riskier Action & 0.79 & 0.98\\
[1.5ex]
120 & Principal Trust & 0.64 & 0.64\\
[1.5ex]
 & Punishment & 1.00 & \vphantom{4} 1.00\\
[1.5ex]
\addlinespace
 & Riskier Action & 0.92 & 0.98\\
[1.5ex]
150 & Principal Trust & 0.78 & 0.74\\
[1.5ex]
 & Punishment & 1.00 & \vphantom{3} 1.00\\
[1.5ex]
 & Riskier Action & 0.98 & \vphantom{1} 1.00\\
[1.5ex]
180 & Principal Trust & 0.81 & 0.82\\
[1.5ex]
\addlinespace
 & Punishment & 1.00 & \vphantom{2} 1.00\\
[1.5ex]
 & Riskier Action & 0.98 & 1.00\\
[1.5ex]
210 & Principal Trust & 0.88 & 0.90\\
[1.5ex]
 & Punishment & 1.00 & \vphantom{1} 1.00\\
[1.5ex]
 & Riskier Action & 1.00 & \vphantom{1} 1.00\\
[1.5ex]
\addlinespace
240 & Principal Trust & 0.94 & 0.96\\
[1.5ex]
 & Punishment & 1.00 & 1.00\\
[1.5ex]
 & Riskier Action & 1.00 & 1.00\\
[1.5ex]
\bottomrule
\end{tabular}
\end{table}

\end{document}
